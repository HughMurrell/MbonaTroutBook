
\chapter{History of Fishing at Mbona}

\section{The Vision}

Awesome Mbona! We, members of the Mbona Family, which extends over 50 years, 
have so much to be grateful and thankful for to a special few, headed up by Eric and Pat MacKenzie. 
What a profound vision they conceptualised, to build the paradise that we have all shared. 
Mbona has had a huge effect on our families and our children who have been able to free themselves 
of fear and many restrictions that have become a way of life in our country.
It is my pleasure and privilege to capture for you the history of trout and fishing as an additional recreation 
available to us on Mbona.

It all started one sunny day in late 1973 as Eric MacKenzie was walking close by the Crystal stream 
accompanied by a friend Edward Austin. Eric asked if there were any signs of fish in the area, 
Edward?s response was to catch a grasshopper and toss it into the stream. 
Almost instantaneously there was a swirl on the surface and the grasshopper was history, 
yes history in the making of trout fishing available to us. 

Such a simple beginning to a sport that has given so much joy to many.
Dams were mapped and designed with details of location demarcated. 
Drummond and Michael MacKenzie were tasked with naming each dam, 
they came up with the names that have become so familiar to us over the years, 
each having a special place in the hearts of enthusiastic anglers. 

They are:  Crystal, Rainbow, Laughter, Amber, Pateric, Emerald, Deep Pool, Evergreen and Holbeck. 
The list was sent to Pat and Eric, who were most impressed with the names. 
Pat was puzzled by the name which she pronounced Paterric for a short time until 
the penny dropped and she and Eric are forever remembered by the special name of this dam.

Today, it is so easy to name the dams, choose where we would like to fish and go 
and enjoy an experience that includes a magnificent environment, peace and quiet, birds, buck 
and occasional greetings to other shareholders as they walk or ride the many routes offered in our paradise. 
What we do not appreciate is the mammoth task that was undertaken to create these facilities. 
Crystal has a dam wall that is over 200meters long, how much earth works was needed is 
mind boggling and the cost in today?s standards would make it prohibitive to build. 
Each dam has been skilfully engineered and apart from Emerald, 
which has an irreparable leak, are functional all the year around.

\section{The Dams}

Some facts about Mbona dams:

\subsection*{Crystal,  21 hectares, 52 acres Completed in December 1972.}

Crystal is our largest dam. In the late 1976 a major rain storm occurred in and around Mbona. 
This resulted in a neighbouring dam above Evergreen, which had Bass in it, to overflow into Evergreen 
and to introduce Bass into its waters. It was decided to drain Evergreen which at that time was 
successful in getting rid of the Bass. Evergreen communicates with Crystal and the bass arrived 
and multiplied at a significant rate. The problem of the Bass in Crystal was the subject of much concern and debate. 
Draining of Crystal was not a consideration. After some time Bass were again seen in Evergreen. 
Today the consensus opinion within the Fishing Subcommittee is to rather create another facility for 
recreation so we have opened Crystal to a limited Bass fishing opportunity. 
Bass fishing is allowed with fixed spool reels and single hooked lures. Catch and keep is what is encouraged. 
This in fact has been a great boon to younger children, who now spend hours trying to catch bass and Bluegill 
which have also proliferated in Crystal. There have been no complaints received and in fact there have 
been a number of Bass caught between 3.5 and 5kg. Trout are caught in Crystal and one?s impression 
is that these numbers are on the increase which we believe is due to a more aggressive stocking program. 
Over the years there has been reference to concerns about reed encroachment in the southern end or inflow area between 
Laughter and Crystal. This is being discussed again, although it is a daunting task to dredge the area because 
silting up is the problem with the reeds flourishing in the enriched shallow water. 
There are discussions re creating a small wetland between Laughter and Crystal.

\subsection*{Evergreen,  10 acres. Completed 1975.}

Referred to above is the Bass problem in Evergreen and the floods in 1976. As noted Evergreen was drained 
and Bass exterminated, this was successfully done with people being asked to get rid of Bass by taking them 
home and trout were netted and transferred to Crystal.  Bass were back in Evergreen shortly after damage to 
the communication channel between Evergreen and Crystal had occurred. This was thought to be due to the 
fact that the Bass were able to swim up the damaged runway. Evergreen also had a problem with weed overgrowth 
and Grass Carp were introduced with good effect. There are no records relating to this, however, despite 
the fact that Grass Carp are supposed to concentrate their efforts on the periphery, the end result was positive. 
Of interest is that at one stage off Ernie?s walk, that circumvents Evergreen, a viewing platform was constructed 
5 meters above the water from which one could watch the fish swimming in the shallows. 
Grass Carp are now classed as exotic and it is virtually impossible to get hold of. Evergreen is a popular dam 
with floating tubes allowed, there are reports of excellent catches including some 3.5 kg bass. 
For some years it  was a catch and release dam, the fishing Committee at that time expressed their satisfaction 
with the outcome of this, however, other fishermen had had years of happy catching and preparing for the 
table and the decision to release has been recently lifted. If only our returns were more reliable, 
we could make a better than guessing as to the results.

\subsection*{Amber, 1 acre and Pateric, 2.5 acres, were both completed in 1973}

These two dams have regular reports of good catches. The recent blockage of the outlet from Pateric to Amber 
has been cleared and Amber has been overflowing for a good part of the year. Pateric has had leaks 
repaired on a number of occasions, however, every winter during the dry season, there is a significant drop in its level. 
A definite observation is that this dam is one of the most productive for catching trout and there is speculation that 
the nutrients available are due to enhanced growth of foliage during the low level periods.
Over the years Amber has produced a number of surprises of Rainbows over 2.5kg and a lone Brown of 2.2kg, 
this fish was returned to the water in good condition.

\subsection*{Rainbow: 3.2 acres, Laughter 4.2 acres, Deep Pool .75 acre and Holbeck 0.25 were all completed before the end of 1975.}

Rainbow has also had its catch and release restriction lifted, it is a very popular productive dam with a recent report of a 
2.5kg Brown caught and released.
Laughter has produced its share of excitement over the years. During the recent past there are no reports of Bass in this dam, 
although Bluegill are reported intermittently, however, 5 and 8 years ago there were Bass in Laughter. In both cases Laughter 
was drained and we have successfully rid  them of bass. There are a number of hypotheses mooted to explain this as there is 
no source of water that could be blamed as the source of the contamination. This included the possibility of fertile eggs attaching 
to the legs of water birds whilst visiting Crystal and then being transferred to Laughter. Our thoughts were more directed to 
suspect some human mischief. Since the last drainage, we have had no reports of Bass.
Each dam is unique and has a special attraction for a number of different fishermen. Both Holbeck and Deep Pool have been 
productive happy places for many of our shareholders. Holbeck is posing a problem with having silted up considerably. 
There are two camps, one wanting it to be left alone and the other hoping to dredge it and return it to its glory as a popular spot.

\section{The Hatchery}

The dams were eventually completed and we had the waters within which we could pursue our desire to enjoy fly fish for trout on Mbona.
The next step was to tackle the project of locating a source for trout, this led to the need to get in expert advice and guidance. 
This required a well thought out plan taking into consideration all the unique elements pertinent to Mbona. 
It was obvious that this was not just going to be an occurrence. It was going to be a long process that needed 
conscientious dedicated people to drive and maintain it. 
It needed financing and the blessing of the Joint Board of Directors who would support it financially and allow access to 
the permanent staff. This management team needed to be replicated as time went on and as the Board changed new 
people were needed to get involved. Throughout our history, we have had these people, 
many of them giving significant input of time and expertise. 

In December of 1973, the first joint board AGMeeting occurred. It was at this meeting that guidance from Terry Oatley 
was sought with respect to trout fishing at Mbona. Terry Oatley was a member of the Natal Parks Board and fortuitously 
was a shareholder at Mbona and this was one of his many contributions.
It was concluded that the location of Mbona, its elevation and its Climate would at best leave us with a situation that 
was marginal in our quest for creating a successful sporting and recreational facility 
for any of our interested shareholders. The initial decision was to stock our dams with 10inch fish. 
This was thought to be the most viable size to allow for optimal survival against weather conditions, predators such as otters, 
fish eagles and cormorants, the numbers of trout stocked in each dam was carefully worked out according to the volume of water 
in each facility, this is still being used as a guide today. Using Terry?s contacts, the Natal Parks Board documented a plan for Mbona. 
It was soon realised that the transport of the larger fish became a major operation which was both expensive and risky with 
quite a high mortality rate. The decision was made to buy 3000 6-7inch trout which had been calculated as a reasonable amount 
to stock our dams and support the fishing demands.

This meant that a facility needed to be built as rearing tanks for the fingerlings to grow them to the ideal stocking size of about 10inches. 
The basic plan was drawn up and this plan has evolved over the years to our present adequate and efficient system. 
There is a gravitational water supply from Lake Crystal to the ponds, which has had to be modified over the years. 
Some of the points are worth mentioning. We have developed this supply over the past 15 years so that it retains its original 
gravitational principle but so that it is adequate, we have created a dual system. This fulfils two essential functions, an adequate supply, 
which is at an appropriate depth to access the lowest water temperature, without being at too much risk of clogging up with 
debris and secondly is to have an adequate back up system. The fact is that the supply has been interrupted at least twice 
in the past with devastating results of wiping out the whole stock in the ponds. We have a rotational system in place where the inlets 
to these water supplies are checked regularly as part of our maintenance program. The quality of the water is enhanced by passing 
it through our aeration tower, which has also been modified at regular intervals. The new system also gives us the choice of 
increasing our water flow through the system, which is needed in the hotter months. We therefore have a quality, 
reliable and self-powered system which will be described in detail in the chapters that follow.

\section{Record Keeping}

One of the most significant developments in the last 10 years is the weighing and counting of the trout at the various stages of 
their lives in the hatchery. The handling has been perfected as are the methods of moving them from the ponds into the 
transporting vehicle, which is fitted with a tank that is well aerated and can accommodate 200 10 inch fish. 
This has allowed us to stock our own dams and deliver to customers, fish that are not compromised and mostly 
don?t even need any resuscitating. This is in contrast to previous efforts that always had a mortality rate. 
We have ensured a process that allows us to accurately size the fish for customers and also for record 
purposes of what is stocked and where.

 Record keeping of expenses has improved, accurate graphs on growth rate and feeding volumes is being 
 more accurately developed and maintained which will allow us to continue improving our production and 
 performance functionally and economically.
 
 The age old problem of monitoring once the fish have left the hatchery continues unabated. 
 From the beginning it was recognised the importance of outcomes of our efforts, fish returns, numbers of 
fishermen, the impact of catch and release initially and further down the line, 
the introduction of float-tubes and allowing them access now to Laughter as well as Crystal and Evergreen. 
Without this data, we will continue to make uneducated guesses rather than produce facts.

From the outset, fish returns have been notoriously poor. There have been a number of unsuccessful schemes, 
including sending a form with the monthly accounts to every shareholder. Apart from the new electronic accounts system 
introduced, this was not a success. Returns to the clubhouse, or the gate on leaving and now by email or on to the website 
are still poorly supported, we would appreciate any suggestions to add a catalyst to this process. 
The truth is, we are not able to assess the success of what we have done without appropriate data.


\section{Predators}
What has been shared is the development of the facility over the years to accommodate decisions made as part of the journey. 
To conclude this aspect, the original plan was well thought out and has not changed much, including the pitfalls to a perfect system. 
There have been a number of threats, apart from physical resource maintenance and development. 
The fish are vulnerable to predators even in the confines of the hatchery, shade cloth and nets are there to protect from birds, 
an electrified fence has been the most effective deterrent against otters, although the fence itself still needs attention to its strength. 
We still have a real, although mysterious reduction in numbers of fish from the raceway. It is our suspicion 
that there may be a human element to this phenomenon and the next step is to monitor by camera more frequently. 

\section{Acknowledgements}

As previously mentioned, the process was made possible by enthusiastic and interested individuals from conception 
to where we are today. Sharing with those involved, the success is always affected by the attitude of the joint board, 
the availability of funds, the management and the support of the trout and fishing committee. 
Over the years, we have been blessed by having these people to contribute to what we have today. 
A common thread commenting on the support and expertise of different managers every step of the way is 
evident and the present situation is better than any of us have experienced previously.

What a long exciting and productive journey this has been spanning fifty years. 
At the risk of not mentioning all the contributors to the epic, people who have been involved and made 
contributions are included as we summarise the progress over the years. Kindly accept apologies for any omissions. 

To Eric and Pat for their vision and the creation of the vast and superb infrastructure, including the dams, 
their beauty and their accessibility, the MacKenzie clan as well are the founders and to whom we will all be eternally enriched and grateful.

Over the years we need to acknowledge the invaluable support and contribution made by the respective Joint Board members. 
The services of managers, without exception, has been remarkable and appreciated by the trout committees. 
The original plan was researched by Ted Oatley, who used his contacts with the Natal Parks Board to produce a detailed assessment 
of the situation and then put together a plan of action. The basics of that plan are still being used today. 
We are in a process and this plan has been modified and developed to suit the needs of each successive committee.

The initial phase was to create the hatchery with the water supply from Crystal. The aerator was built, the three ponds created 
which were originally made of corrugated iron and the raceway. Garth Hatton was the first trout committee chairman. 
His services extended over the next 15 years, he has made a fantastic contribution over these years and was 
faced with many problems and also contributed to modifying the original plans to improve, not only the physical resources, 
but also the quality of the actual fishing. 

During his time he faced many challenges which included making the hatchery safer from predators, 
draining of Evergreen to get rid of the bass, increasing the capacity of the ponds as well as converting the ponds to brick and mortar, 
he rebuilt the aeration tower and produced a number of comprehensive reports which related to quality of the water in the dams, 
fishing rules and advice on stocking. He was well supported by the board and particularly the managers.

Chairmen that followed were among others, Richard Erasmus and Ken Cohen who was followed by 
first Nick and then Charles Shave who together served for more than 10 years up until 2005. 
All these men made every effort to maintain and improve conditions, each one has faced challenges including 
some disasters where large numbers of fish died. 
This was due to water supply failure, inclement weather conditions, otter invasions and even a few incidents of poaching. 
They were party to the decision to create catch and release restrictions in Rainbow and Evergreen, 
they retained the sourcing, purchasing and growing the trout, they were responsible for maintaining the hatchery 
and for seeing that the dams were properly stocked. They were continually devoted to improving the fishing as a quality recreation.

During the years, we have had expert advice and input from well known and respected trout fundies particularly 
Rob Karssing and Jake Alletson are acknowledged and thanked.

The present committee has been functional since 2005. They have faced all the problems of their predecessors 
including episodes of losses due to water blockages, maintenance of the hatchery and trying in vain to 
get better records of number of fishermen and their catches. An important decision was taken to buy trout eggs 
instead of buying fingerlings of 6 inches. This decision was taken in 2012 and it has changed our process 
considerably and has allowed us to increase our numbers substantially. The resources have been documented 
earlier in this article as well as the process of rearing the trout through the various phases of development. 
The venture has been successful.  We researched the process carefully and introduced scientific measures to 
measure temperature and oxygen content in the water accurately. 

We have improved the management of moving of trout and have got this down to a fine art. 
There has also been an improvement in all record keeping both monitoring growth and general running costs. 
The driving principle was to improve the fishing at Mbona primarily and secondly to sell fish of 30cms in size to neighbouring facilities. 
This has proved quite successful and we will continue to have frequent meetings, hoping to improve all of our ventures. 
We also took the decision to open Crystal to limited bass fishing as previously mentioned and to allow floating tubes on Laughter. 
The latter decision resulted in an exciting afternoon?s fishing producing over 20 nice sized fish. 
The committee is very active and meets regularly and would welcome new members, especially some younger enthusiasts.

There are a number of issues that we would really appreciate help with and suggestions one of  these is record keeping, despite the website facility, still leaves much to be desired. This is so important for us to know whether we are being successful or not and will guide in future plans. The Easter and Christmas fishing competitions are not as well supported as we would like. We are planning to supply trout to shareholders by an ordering system and would like to hear whether there is enough of a demand for them.

In conclusion, all previous contributors to this facility are to be saluted and thanked. 
The present committee has been active since 2005. It has been a great journey together of enthusiastic members 
doing research and development. This has been a shared experience. Each member has brought to the 
committee something special and all have been willing to attend meetings and especially to carry out tasks that they have undertaken to do. 

Jacko Jackson brought a wealth of scientific know how and expertise, 

Pierre Olivier has been a tower of strength with his practical experience and unrelenting service of maintenance, 
development and even deliveries is invaluable. 

Bernard McDonald always willing, contributed much knowledge and was always prepared to be hands on. 

Pete Barbour was our marketer and has established a nice clientele which has now been taken over by Ronnie Ritchie who is already producing results. 

It has been my privilege to work with this fine group of people. It must be said that Gareth Powell and Dave Forsyth have given us every support and assistance whenever we have asked and have made their contributions to what we have. 

We thank the boards over the years for their support and guidance. We believe that Mbona can be proud of our facility. 
We have as a first class entity that is functional and productive and that will continue to bring to shareholders a treasured resource. 
Any shareholders are welcome to come and see what we have in the hatchery and to be further informed of the process. 
We need some of our younger members to join us and take over.

We wish you great moments as you enjoy the unique and precious delights of Mbona and that you will at times be rewarded with a Rainbow trout of note.


{\bf Dennis Dyer}

{\bf Fishing Committee Chairman}












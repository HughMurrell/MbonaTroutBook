\chapter{Mbona Hatcheries Records}


\section{Record Keeping}

The various record keeping forms appear in the appendix of this document. The forms are self explanatory.
Eventually the Appendix will also contain records from years gone by if they still exist. 

\subsection{Season starting June 2018}

The 2018 season started with the arrival of $10000$ eyed Rainbow Trout eggs from the Underberg Hatchery. 
The temperature of the water at the Underberg hatchery was approximately \SI{12}{\celsius}.
The eggs arrived at Mbona at14h30 in two trays with one ice tray on the bottom and 
another ice tray on the top of the transportation cooler box.

On arrival at Mbona the temperature of the eggs in the top tray was \SI{12}{\celsius} which was brought up to
the Mbona water temp of \SI{14.6}{\celsius} by 16h00 and these eggs were then moved to bath B.

On arrival at Mbona the temp of the bottom tray was \SI{10}{\celsius} 
which was brought up to Mbona water temp of \SI{14.6}{\celsius} by 16h30 and these eggs were placed
in bath A.

Once the eggs had been transferred to the incubation baths, the process of removing dead eggs began.
In table ~\ref{tab:Incubation2018} we show how many dead eggs were removed from each bath per day
during the incubation period.

Feeding the Rainbow fry with fish food starter powder commenced on Tuesday the $12^{th}$ of June.

On Thursday the $14{th}$ of June a further $3000$ brown trout eyed eggs from the Trova Trout company 
in Sabie arrived at the Mbona Hatchery. These ova were packed in iced trays and air freighted to PmB. 
They arrived at Mbona at 9am at a temperature of \SI{5.1}{\celsius}. 
The temperature of the eggs was gradually increased to \SI{13.1}{\celsius} over a period of 5 hours. 
The ova were transferred to a hatching tray in bath C at 14h00 when the water was at 
temperature \SI{13.4}{\celsius}. 
Removal of dead ova from Bath C commenced on Friday the $15^{th}$ of June, 
see table ~\ref{tab:Incubation2018} for daily mortality counts.

\begin{table}[H]
  \centering
  \includegraphics[scale = 0.9]{tables/TablesIncubationMortality.pdf}
   \caption{Temperature and mortality records for 2018 incubation period.}
  \label{tab:Incubation2018}
\end{table}

Assuming that the number of eggs at purchase was accurate and that dead eggs were counted accurately
table ~\ref{tab:FryRelocation2018} suggests that we could expect approximately 5000 surviving Rainbow fry 
ready for relocation to the fry tanks. So we were pleasantly surprised when the Relocation counts as 
shown in table ~\ref{tab:FryRelocation2018} indicated approximately 8000 surviving Rainbow fry.

\begin{table}[H]
  \centering
  \includegraphics[scale = 0.9]{tables/TablesFryRelocationRecord.pdf}
   \caption{Counts for 2018 relocation of fry from baths to tanks.}
  \label{tab:FryRelocation2018}
\end{table}

This 40\% discrepancy is either due to receiving more eggs than expected or due to the
counting of hatched egg shells as dead eggs or to both.

\subsubsection{Rearing the fry in the tanks}
Two weeks after relocation of fry from baths to tanks we carried out length and weight measurements
on the fry. The results can be seen in table ~\ref{tab:FryGrowth} where the recommended feeding
schedule is given as computed from the suggestions given in chapter 3.

\begin{table}[H]
  \centering
  \includegraphics[scale = 0.9]{tables/TablesFryGrowth.pdf}
   \caption{Weight and Length measurements of growing fry}
   \label{tab:FryGrowth}
\end{table}

\subsection{Sales of 2017 Trout in the 2018-2019 season}

During the 2018-19 season we sell 2017 fish live to local customers and 
when available, dressed fish to Mbona shareholders. We use the remainder
to stock our dams at Mbona. see tables, \ref{tab:ExternalSales2018}, 
 and \ref{tab:MbonaDamSales2018} below.

\begin{table}[H]
  \centering
  \includegraphics[scale = 1.2]{tables/TablesExternalSales.pdf}
   \caption{2018-19 sales of live fish to external customers.}
  \label{tab:ExternalSales2018}
\end{table}


\begin{table}[H]
  \centering
  \includegraphics[scale = 1.2]{tables/TablesMbonaDamSales.pdf}
   \caption{2018-2019 stocking of Mbona dams from eggs hatched in 2017.}
  \label{tab:MbonaDamSales2018}
\end{table}

Since October 2018 we have started recouping some of our costs by selling both dressed 
and smoked trout to shareholders,
see chart \ref{fig:ShareholderSales} below.

\begin{figure}[H]
  \centering
  \includegraphics[scale = 0.75]{tables/MbonaTroutShareholderSales.pdf}
   \caption{Nov 2018 - May 2019 sales of dressed and smoked trout to Mbona shareholders.}
  \label{fig:ShareholderSales}
\end{figure}

\subsection{Budget proposal}

Based on the sales figures presented above and food and labour expenses made available
by the Mbona management we propose the following rudimentary budget for the 2019-2020
season. This proposal would benefit from further deliberations by the fishing committee.

\begin{table}[H]
  \centering
  \includegraphics[scale = 0.9]{tables/TablesBudget.pdf}
   \caption{Possible Budget proposal for 2019-2020 season.}
  \label{tab:Budget}
\end{table}
 

\subsection{Fly Fishing returns}

The primary purpose for the trout rearing operation is to provide recreational
fly fishing for shareholders. It is difficult to gauge how popular fly fishing is. 

We encourage shareholders
and their guests to fill in the Mbona trout return form and tell us how many trout they have
caught and in what dam they were successful. Your returns can be submitted online
by visiting the Mbona website: 

\url{www.mbona.co.za} 

or if you are averse to using a computer you can fill in a paper based form at the Mbona gate.

In the next two charts the reader can find an analysis of trout returns since the start of
web-based record keeping.

\subsection{Mbona Trout Ladder}

\begin{figure}[H]
\centering
  \includegraphics[scale=0.4]{tables/MbonaTroutLadder.pdf}
   \caption{Ladder showing catch and keep records for each share since web based trout returns were introduced.}
  \label{fig:MbonaTroutLadder}
\end{figure}


\subsection{Trout Catches by Dam}

\begin{figure}[H]
\centering
  \includegraphics[scale=0.4]{tables/TroutCatchesByDam.pdf}
   \caption{Bar chart showing catch and keep records for each dam for 2018.}
  \label{fig:TroutCatchesByDam}
\end{figure}



